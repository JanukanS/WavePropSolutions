\documentclass[11pt]{article}
\usepackage{amsmath}
\usepackage{tikz}
\title{Another Approach To Computing Electromagnetic Ray Trajectories Within Plasma Varying In 2 Dimensions}
\date{}
\begin{document}


\maketitle
\section*{Approximation of an electron density distribution with 2D linear basis functions}
Given a distribution of electron density values defined on rectangular nodes, the electron density can be interpolated within a single rectangular cell using 2D linear basis functions. The following expression describes an interpolation based on 2D linear basis functions acting on a rectangular cell of width $\alpha$ and height $\beta$
\newline

%\begin{tikzpicture}

%\fill[blue!40!white] (0,0) rectangle (5,5);
%\filldraw[black] (0,0) circle (2pt) node[anchor=west] {n1};
%\filldraw[black] (0,0) circle (2pt) node[anchor=west] {Intersection point};
%\filldraw[black] (0,0) circle (2pt) node[anchor=west] {Intersection point};
%\filldraw[black] (0,0) circle (2pt) node[anchor=west] {Intersection point};
%\end{tikzpicture}
\begin{equation}
n_e = \sum^4_{i=1}{n_{i}\Phi_i(x,y)}
\end{equation}
\begin{align}
\Phi_1(x,y) &= \bigg(1-\frac{x}{\alpha}\bigg)\bigg(1-\frac{y}{\beta}\bigg) \tag{2a}\\
\Phi_2(x,y) &= \bigg(\frac{x}{\alpha}\bigg)\bigg(1-\frac{y}{\beta}\bigg) \tag{2b}\\
\Phi_3(x,y) &= \bigg(1-\frac{x}{\alpha}\bigg)\bigg(\frac{y}{\beta}\bigg) \tag{2c}\\
\Phi_4(x,y) &= \bigg(\frac{x}{\alpha}\bigg)\bigg(\frac{y}{\beta}\bigg) \tag{2d}
\end{align}

Expanding the basis functions results in the following expression for electron density
\begin{equation*}
\operatorname{n_{e}}{\left(x,y \right)} = n_{1} + x y \left(\frac{n_{1}}{\alpha \beta} - \frac{n_{2}}{\alpha \beta} - \frac{n_{3}}{\alpha \beta} + \frac{n_{4}}{\alpha \beta}\right) + x \left(- \frac{n_{1}}{\alpha} + \frac{n_{2}}{\alpha}\right) + y \left(- \frac{n_{1}}{\beta} + \frac{n_{3}}{\beta}\right)
\end{equation*}
\section*{Hamilton's Ray Equations for Electromagnetic Rays}
The following equations are the Hamilton's ray equations for electromagnetic rays within plasma. 
\begin{equation}
\frac{dx}{d t}  = \frac{c^{2} \operatorname{k_{x}}}{\omega}
\tag{3}
\end{equation}
\begin{equation}
\frac{dk_x}{d t} = \frac{e^2 }{m \epsilon_0}\frac{dn_e}{dx}
\tag{4}
\end{equation}
\begin{equation}
\frac{dy}{d t}  = \frac{c^{2} \operatorname{k_{y}}{}}{\omega}
\tag{5}
\end{equation}
\begin{equation}
\frac{dk_y}{d t} = \frac{e^2 }{m \epsilon_0}\frac{dn_e}{dy}
\tag{6}
\end{equation}
The trajectory of an electromagnetic ray can be solved with initial values for $x$,$y$,$k_x$,$k_y$, referenced as $x_o$,$y_o$,$k_{x,o}$,$k_{y,o}$. For simplification the following expression is defined
$$\gamma = \frac{e^2}{m\epsilon_0}$$

\section*{Analytical Solution}
The following expressions describe the analytical trajectories, making use of several constant coefficients that have values based on the corner density values and the initial coefficients. 

\begin{equation}
x{\left(t \right)} = - \frac{D_{1}}{D_{2}} + E_{1} e^{- \sqrt[4]{D_{2}} t} + E_{2} e^{\sqrt[4]{D_{2}} t} + E_{3} \sin{\left(\sqrt[4]{D_{2}} t \right)} + E_{4} \cos{\left(\sqrt[4]{D_{2}} t \right)} \tag{7}
\end{equation}
\begin{equation}
y{\left(t \right)} = - \frac{D_{3}}{D_{4}} + F_{1} e^{- \sqrt[4]{D_{4}} t} + F_{2} e^{\sqrt[4]{D_{4}} t} + F_{3} \sin{\left(\sqrt[4]{D_{4}} t \right)} + F_{4} \cos{\left(\sqrt[4]{D_{4}} t \right)}
\tag{8}
\end{equation}
\section*{Application}
This analytical solution could be used for numerical solutions to EM ray trajectories. The computed rays would consist of segments of analytical solutions chained together, computed by the process below:
\begin{enumerate}
\item Compute some constant coefficients based on electron density sampling points at the corner of each rectangular cell.
\item Define initial conditions for an EM ray
\item Based on the rectangular cell that the EM ray is in, compute the remainder of the constant coefficients
\item With the analytical solution complete with coefficients, use root-finding algorithms to determine the time it will take for a ray to hit a cell boundary
\item Store the coefficients along with the time in cell before collision with boundary
\item Return to step 3 with new rectangular cell until simulation time is complete
\end{enumerate}
This approach may be applicable to other basis functions i.e. triangular meshes or 3D meshes.
\section*{Constants}
\subsection*{C coefficients}
\begin{align*}
C_{1} &= \frac{c^{2} \gamma \left(- \frac{n_{1}}{\alpha} + \frac{n_{2}}{\alpha}\right)}{2 \omega^{2}}
\tag{9a} \\
C_{2} &= \frac{c^{2} \gamma \left(\frac{n_{1}}{\alpha \beta} - \frac{n_{2}}{\alpha \beta} - \frac{n_{3}}{\alpha \beta} + \frac{n_{4}}{\alpha \beta}\right)}{2 \omega^{2}}
\tag{9b} \\
C_{3} &= \frac{c^{2} \gamma \left(- \frac{n_{1}}{\beta} + \frac{n_{3}}{\beta}\right)}{2 \omega^{2}}
\tag{9c} \\
C_{4} &= \frac{c^{2} \gamma \left(\frac{n_{1}}{\alpha \beta} - \frac{n_{2}}{\alpha \beta} - \frac{n_{3}}{\alpha \beta} + \frac{n_{4}}{\alpha \beta}\right)}{2 \omega^{2}} \tag{9d} \\
\end{align*}
\subsection*{D coefficients}
\begin{align*}
D_{1} &= \frac{c^{4} \gamma^{2} \left(- \frac{n_{1}}{\beta} + \frac{n_{3}}{\beta}\right) \left(\frac{n_{1}}{\alpha \beta} - \frac{n_{2}}{\alpha \beta} - \frac{n_{3}}{\alpha \beta} + \frac{n_{4}}{\alpha \beta}\right)}{4 \omega^{4}}
\tag{10a} \\
D_{2} &= \frac{c^{4} \gamma^{2} \left(\frac{n_{1}}{\alpha \beta} - \frac{n_{2}}{\alpha \beta} - \frac{n_{3}}{\alpha \beta} + \frac{n_{4}}{\alpha \beta}\right)^{2}}{4 \omega^{4}}
\tag{10b} \\
D_{3} &= \frac{c^{4} \gamma^{2} \left(- \frac{n_{1}}{\alpha} + \frac{n_{2}}{\alpha}\right) \left(\frac{n_{1}}{\alpha \beta} - \frac{n_{2}}{\alpha \beta} - \frac{n_{3}}{\alpha \beta} + \frac{n_{4}}{\alpha \beta}\right)}{4 \omega^{4}} \tag{10c} \\
D_{4} &= \frac{c^{4} \gamma^{2} \left(\frac{n_{1}}{\alpha \beta} - \frac{n_{2}}{\alpha \beta} - \frac{n_{3}}{\alpha \beta} + \frac{n_{4}}{\alpha \beta}\right)^{2}}{4 \omega^{4}} \tag{10d} \\
\end{align*}

\subsection*{E coefficients (x solution)}
\begin{align*}
E_{1} &= - \frac{C_{2} c^{2} k_{yo}}{4 D_{2}^{0.75} \omega} + \frac{D_{1}}{4 D_{2}} + \frac{x_{o}}{4} + \frac{C_{1} + C_{2} y_{o}}{4 \sqrt{D_{2}}} - \frac{c^{2} k_{xo}}{4 \sqrt[4]{D_{2}} \omega} \tag{11a}\\
E_{2} &= \frac{C_{2} c^{2} k_{yo}}{4 D_{2}^{0.75} \omega} + \frac{D_{1}}{4 D_{2}} + \frac{x_{o}}{4} + \frac{C_{1} + C_{2} y_{o}}{4 \sqrt{D_{2}}} + \frac{c^{2} k_{xo}}{4 \sqrt[4]{D_{2}} \omega} \tag{11b}\\
E_{3} &= - \frac{C_{2} c^{2} k_{yo}}{2 D_{2}^{0.75} \omega} + \frac{c^{2} k_{xo}}{2 \sqrt[4]{D_{2}} \omega}
\tag{11c}\\
E_{4} &= \frac{D_{1}}{2 D_{2}} + \frac{x_{o}}{2} - \frac{C_{1} + C_{2} y_{o}}{2 \sqrt{D_{2}}} \tag{11d}\\
\end{align*}
\subsection*{F coefficients (y solution)}
\begin{align*}
F_{1} &= - \frac{C_{4} c^{2} k_{xo}}{4 D_{4}^{0.75} \omega} + \frac{D_{3}}{4 D_{4}} + \frac{y_{o}}{4} + \frac{C_{3} + C_{4} x_{o}}{4 \sqrt{D_{4}}} - \frac{c^{2} k_{yo}}{4 \sqrt[4]{D_{4}} \omega} \tag{12a} \\
F_{2} &= \frac{C_{4} c^{2} k_{xo}}{4 D_{4}^{0.75} \omega} + \frac{D_{3}}{4 D_{4}} + \frac{y_{o}}{4} + \frac{C_{3} + C_{4} x_{o}}{4 \sqrt{D_{4}}} + \frac{c^{2} k_{yo}}{4 \sqrt[4]{D_{4}} \omega} \tag{12b} \\
F_{3} &= - \frac{C_{4} c^{2} k_{xo}}{2 D_{4}^{0.75} \omega} + \frac{c^{2} k_{yo}}{2 \sqrt[4]{D_{4}} \omega}
\tag{12c} \\
F_{4} &= \frac{D_{3}}{2 D_{4}} + \frac{y_{o}}{2} - \frac{C_{3} + C_{4} x_{o}}{2 \sqrt{D_{4}}} 
\tag{12d} \\
\end{align*}
\end{document}
